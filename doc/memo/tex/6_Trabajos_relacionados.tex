\capitulo{6}{Trabajos relacionados}

En este apartado se exponen tres herramientas relacionadas con la desarrollada en este proyecto, una dedicada a la obtención de datos y dos al análisis.

\section{GHTorrent}

GHTorrent \cite{Gousi13} es una herramienta que obtiene datos desde la API de GitHub y los almacena en dos bases de datos:

\begin{itemize}
	\item MongoDB no relacional, guarda los datos en bruto.
	\item Base de datos relacional con entidades enlazadas.
\end{itemize}

GHTorrent es utilizado por investigadores y empresas, como fuente procesos software y analíticas de productos.

A continuación se muestra una tabla comparativa entre nuestra herramienta y GHTorrent:

\begin{table}[H]
\centering
\begin{tabular}{lcc}
\toprule
Característica & Anvireco & GHTorrent \\
\midrule
Obtención de entidades de GitHub & \cellcolor{green!25} {$\checkmark$} & \cellcolor{green!25} {$\checkmark$} \\
API REST & \cellcolor{green!25} {$\checkmark$} & \cellcolor{red!25} {$\times$} \\
Cliente web & \cellcolor{green!25} {$\checkmark$} & \cellcolor{red!25} {$\times$} \\
Análisis visual & \cellcolor{green!25} {$\checkmark$} & \cellcolor{red!25} {$\times$} \\
Base de datos NoSQL & \cellcolor{green!25} {$\checkmark$} & \cellcolor{green!25} {$\checkmark$} \\
Base de datos SQL & \cellcolor{red!25} {$\times$} & \cellcolor{green!25} {$\checkmark$} \\
Ejecución distribuida & \cellcolor{red!25} {$\times$} & \cellcolor{green!25} {$\checkmark$} \\
\bottomrule
\end{tabular}
\caption{Comparativa de las características de Anvireco y GHTorrent.}
\label{comparativa-anvireco-ghtorrent}
\end{table}

\section{ReDA - Review Data Analyzer}

ReDA \cite{reda2014} es una aplicación web para la visualización del proceso de revisiones en software de código libre. Para la creación de gráficos hace uso de la librería JavaScript D3.js.

Actualmente permite visualizar el histórico de revisiones del repositorio Android Open Source Project basado en Gerrit.

A continuación se muestra una tabla comparativa entre nuestra herramienta y ReDA:

\begin{table}[H]
\centering
\begin{tabular}{lcc}
\toprule
Característica & Anvireco & ReDA \\
\midrule
Trabaja con datos de GitHub & \cellcolor{green!25} {$\checkmark$} & \cellcolor{red!25} {$\times$} \\
Múltiples repositorios & \cellcolor{green!25} {$\checkmark$} & \cellcolor{red!25} {$\times$} \\
Permite solicitar repositorios & \cellcolor{green!25} {$\checkmark$} & \cellcolor{red!25} {$\times$} \\
API REST & \cellcolor{green!25} {$\checkmark$} & \cellcolor{red!25} {$\times$} \\
Descarga de datos en CSV & \cellcolor{green!25} {$\checkmark$} & \cellcolor{green!25} {$\checkmark$} \\
Gráficos sobre revisiones & \cellcolor{green!25} {$\checkmark$} & \cellcolor{green!25} {$\checkmark$} \\
Gráficos sobre usuarios & \cellcolor{green!25} {$\checkmark$} & \cellcolor{green!25} {$\checkmark$} \\
\bottomrule
\end{tabular}
\caption{Comparativa de las características de Anvireco y ReDA.}
\label{comparativa-anvireco-reda}
\end{table}

\section{GiLA - GitHub Label Analyzer}

GiLA \cite{gila} es una herramienta de visualización que permite el análisis del uso de etiquetas (frecuencia, relaciones, etc) en un proyecto de GitHub.

También permite analizar el modo en que los usuarios utilizan las etiquetas.

A continuación se muestra una tabla comparativa entre nuestra herramienta y GiLA:

\begin{table}[H]
\centering
\begin{tabular}{lcc}
\toprule
Característica & Anvireco & ReDA \\
\midrule
Trabaja con datos de GitHub & \cellcolor{green!25} {$\checkmark$} & \cellcolor{green!25} {$\checkmark$} \\
Múltiples repositorios & \cellcolor{green!25} {$\checkmark$} & \cellcolor{green!25} {$\checkmark$} \\
Permite solicitar repositorios & \cellcolor{green!25} {$\checkmark$} & \cellcolor{red!25} {$\times$} \\
API REST & \cellcolor{green!25} {$\checkmark$} & \cellcolor{red!25} {$\times$} \\
Descarga de datos en CSV & \cellcolor{green!25} {$\checkmark$} & \cellcolor{red!25} {$\times$} \\
Gráficos sobre revisiones & \cellcolor{green!25} {$\checkmark$} & \cellcolor{red!25} {$\times$} \\
Gráficos sobre usuarios & \cellcolor{green!25} {$\checkmark$} & \cellcolor{green!25} {$\checkmark$} \\
\bottomrule
\end{tabular}
\caption{Comparativa de las características de Anvireco y GiLA.}
\label{comparativa-anvireco-gila}
\end{table}