\capitulo{1}{Introducción}

El desarrollo de software es un proceso que siempre se encuentra en constante evolución. Las metodologías existentes se caracterizan por su diversidad y flexibilidad, y es común la aparición de nuevas técnicas y tendencias con objetivos como, por ejemplo, incrementar la calidad del software desarrollado.

Las metodologías de desarrollo ágil y colaborativo actualmente están incorporando, entre otras características, las revisiones de código como vía para crear código de forma más eficiente, con menos defectos, con una mantenibilidad mayor, y en definitiva, mejor calidad.

El concepto de revisión de código es propuesto por M.E. Fagan en 1976 \cite{5388086}, pero su práctica no estaba muy generalizada. Sin embargo, actualmente se puede observar que cada vez son más los proyectos que incorporan las revisiones de código a su ciclo de desarrollo.

GitHub \cite{github:home}, la mayor plataforma de desarrollo colaborativo, ha incorporado a principios de 2017 funcionalidades para realizar revisiones de código a través de su sistema de Pull Requests.

%\imagen{githubFeatures}{Revisiones de código como característica de GitHub.}

Grandes proyectos como Elasticsearch \cite{elastic:elasticsearch} de Elastic o WebFundamentals \cite{google:webfundamentals} de Google ya están utilizando activamente las revisiones de código de GitHub. También existen herramientas para este propósito específico como Gerrit Code Review \cite{gerrit:home} de Google utilizada en proyectos como Android \cite{android:home}, Linux Foundation \cite{linuxfoundation:home} o Eclipse \cite{eclipse:home}.

Para que las revisiones de código sean útiles deben llevarse a cabo por expertos revisores. Actualmente la evaluación y análisis de la calidad de revisiones y revisores es una práctica muy poco común, sin embargo que podría contribuir a una mejora en el proceso de selección de revisores para cada caso concreto, haciendo de la revisión de código un proceso más eficaz y fiable.

\imagen{analisisGraficoRevisor}{Ejemplo de análisis gráfico de un experto revisor.}

Este proyecto se centra en la creación de una herramienta que permita obtener y representar gráficamente datos sobre revisiones de código realizadas en repositorios de GitHub a través de las Pull Requests para facilitar su análisis.

\section{Estructura de la memoria}

La memoria tiene la siguiente estructura:

\begin{itemize}
\tightlist
	\item \textbf{Introducción}: Esta parte de la memoria aborda el contexto del problema y una breve explicación de la solución desarrollada.
	\item \textbf{Objetivos del proyecto}: En esta parte de la memoria se exponen los objetivos perseguidos con la realización del proyecto. 
	\item \textbf{Conceptos teóricos}: Apartado de la memoria donde se exponen una serie de conceptos teóricos relacionados con el proyecto.
	\item \textbf{Técnicas y herramientas}: Esta parte de la memoria está dedicada a la descripción de las diversas técnicas y herramientas empleadas durante el desarrollo del proyecto.
	\item \textbf{Aspectos relevantes del desarrollo del proyecto}: Parte de la memoria que aborda los aspectos más destacables del desarrollo del proyecto.
	\item \textbf{Trabajos relacionados}: Exposición de trabajos relacionados con el proyecto.
	\item \textbf{Conclusiones y líneas de trabajo futuras}: En esta parte de la memoria se detallan las conclusiones obtenidas tras el desarrollo y qué líneas de trabajo se podrían seguir a continuación.
\end{itemize}
